\documentclass[Royal,times,sageh]{sagej}

\usepackage{moreverb,url,natbib, multirow, tabularx}
\usepackage[colorlinks,bookmarksopen,bookmarksnumbered,citecolor=red,urlcolor=red]{hyperref}



% tightlist command for lists without linebreak
\providecommand{\tightlist}{%
  \setlength{\itemsep}{0pt}\setlength{\parskip}{0pt}}



\usepackage{float}


\begin{document}


\setcitestyle{aysep={,}}

\title{canaccessR: An open data product for transit accessibility
analysis in Canada's largest metropolitan areas.}

\runninghead{}

\author{João Figueroa Amorim PARGA\affilnum{}, Anastasia
SOUKHOV\affilnum{}, Robert Nutifafa ARKU\affilnum{}, Christopher
HIGGINS\affilnum{}, Antonio PÁEZ\affilnum{}}

\affiliation{}



\begin{abstract}
In this paper, we describe the \{canaccessR\} package, an open data
product (ODP) created in R that contains public transit travel time
estimates to employment locations and grocery stores across Canada's 12
largest metropolitan areas. We calculate travel time matrices (TTM) from
and to each Dissemination Area (DA) within these regions for the years
2019 and 2023. We add value to the urban analytics community by
processing and integrating raw data, and disseminating user-ready data
in the domain of transportation accessibility in Canada. To do so, we
use the \{r5r\} R package, General Transit Feed Specification (GTFS),
OpenStreetMap (OSM), DMTI's Enhanced Points of Interest, and Statistics
Canada Census data. This data package can be used by researchers,
practitioners, and transit agencies to estimate accessibility levels to
these two essential destinations within these urban areas. Moreover,
travel time matrices are computed from DA centroid to DA centroid, which
means that they can be adapted for use in applications with any type of
destination that is aggregated at the DA level. Finally, as an ODP, the
\{canaccess\} package allows for open exploration, use, and contribution
by users through its GitHub repository.
\end{abstract}

\keywords{Public transit accessibility; open data products (ODPs); R
data package; travel time matrices.}

\maketitle



\bibliographystyle{sageh}
\bibliography{bibfile}


\end{document}
