\documentclass[Royal,times,sageh]{sagej}

\usepackage{moreverb,url,natbib, multirow, tabularx}
\usepackage[colorlinks,bookmarksopen,bookmarksnumbered,citecolor=red,urlcolor=red]{hyperref}



% tightlist command for lists without linebreak
\providecommand{\tightlist}{%
  \setlength{\itemsep}{0pt}\setlength{\parskip}{0pt}}





\begin{document}


\setcitestyle{aysep={,}}

\title{canaccessR: An open data product for analyzing transportation
accessibility to employment and grocery stores in Canada's largest
metropolitan areas.}

\runninghead{}

\author{João Pedro Figueira Amorim Parga\affilnum{}, Anastasia
Soukhov\affilnum{}, Robert Nutifafa Arku\affilnum{}, Christopher
Higgins\affilnum{}, Antonio Páez\affilnum{}}

\affiliation{}



\begin{abstract}
In this paper, we describe the \{canaccesR\} package, an open data
product (ODP) created in R that contains public transit travel time
estimates to employment locations and grocery stores across Canada's 12
largest metropolitan areas. We calculate travel time matrices (TTM) from
and to each Dissemination Area (DA) within these regions for the years
2019 and 2023. We add value to the urban analytics community by
processing and integrating raw data, and disseminating user-ready data
in the domain of transportation accessibility in Canada. To do so, we
use the \{r5r\} R package, General Transit Feed Specification (GTFS),
OpenStreetMap (OSM), DMTI's Enhanced Points of Interest, and Statistics
Canada Census data. This data package can be used by researchers,
practitioners, and transit agencies to estimate accessibility levels to
these two essential destinations within these urban areas. Moreover,
these estimations can be used as inputs in equity and inequalities
assessments, through the comparisons of within and across accessibility
levels found throughout Canada's largest metropolitan areas.
Consequently, we expect to contribute to informed and data-based
decision making in transportation by disseminating these data. We hope
that these datasets can substantiate future improvements in
policy-making that may lead to greater justice in the country's urban
transportation systems. The package is still in its initial phase and
may undergo expansions in the future by adding TTM's for other
destinations (e.g., schools, healthcare facilities). Finally, as an
ODPs, the \{canaccess\} package allows for open exploration, use, and
contribution by users through its GitHub repository.
\end{abstract}

\keywords{Public transit accessibility; open data products (ODPs); R
data package; travel time matrices.}

\maketitle

\section{Introduction}\label{introduction}

The objective of this paper is to describe the \{canaccesR\} open data
package. Its main contents are a set of public transit travel time
matrices (TTM) estimates to employment and groceries stores from the 12
largest Canadian metropolitan areas in 2019 and 2023. The results are
stored at the Dissemination Areas (DA) \footnote{Dissemination Areas are
  the smallest publicly available spatial unit provided by Statistics
  Canada \citep{governmentofcanadaDictionaryCensusPopulation2021a}.}
level, yielding origin-destination pairs containing information on
travel time by public transit, population, and total employment within
each metropolitan area. These estimates were created by leveraging
expertise in data science, computer programming, and transportation
accessibility using the \{r5r\} package
\citep{pereiraR5rRapidRealistic2021}. We used public transit schedule,
transport network, population, and business location data from different
sources to estimate these TTM's datasets.

Estimating accessibility - the potential offered by the transportation
system to reach destinations
\citep{paezMeasuringAccessibilityPositive2012} - requires specialized
datasets and technical expertise. Recent efforts following an
open-source and transparent philosophy have been made to disseminate
useful data and information on transportation in the Canadian context
\citep{soukhovTTS2016RDataSet2023}. However, despite these initiatives,
pre-processed and available data that allow for the ease estimation of
accessibility indicators are still scarce. Within this context, we
expect to help filling this gap by the processing raw into user-ready
data and making them publicly available to advance knowledge on the
field. Our main contribution is to provide analysis-ready data for
Canada's largest cities on the topic of transportation accessibility,
thus making urban analytics in the country more accessible and
contributing to future research and data-based decision making.

The package's main audiences are Canadian researchers in urban planning
and transportation and transportation system agencies. We anticipate
three primary uses for the open data product (ODP) described in this
paper. First, the datasets allow for static assessment of the level of
public transit accessibility across the country's largest cities before
and after the COVID-19 pandemic. In other words, \{canaccesR\} makes it
easier for those interested in comparing cities regarding their level of
public transit accessibility to essential destinations (such as
employment centers and groceries stores) to do so. Second, the temporal
and spatial characters of the datasets made available here allow
researchers to evaluate accessibility changes through time and across
space within the largest Canadian urban areas. Third, as is now common
practice in transportation accessibility research, used as inputs, these
estimates can substantiate broader investigations on transportation
justice and equity
\citep{higginsChangesAccessibilityEmergency2021, humbertoHowTranslateJustice2023, pereiraGeographicAccessCOVID192021}.
For example, the TTM estimates allow for evaluating the evolution of
public transit's accessibility by income or spatial distribution across
all Dissemination Areas (DA's) of each of the 12 cities in the sample
\citep{pargaDemocraticAccessOur2024}. In other words, the package's
contents can be used from straightforward assessments of accessibility
in Canadian urban areas to more theoretically and morally complex
evaluations of justice in the country's urban transportation system.

Besides this introduction, we organize this paper as follows. The next
section contains a description of the data sources we used to construct
the data package. Then, we recount the data processing necessary to
create the package. Next, we go through the main contents of the data
package, i.e., the travel time matrices estimated through our analysis.
We present some basic descriptive statistics of these datasets, and
elucidate how one can use them in accessibility analysis. Finally, we
conclude by explaining how we expect \{canaccesR\} to contribute to the
urban analytics and science community.

\section{Data and methods}\label{data-and-methods}

\subsection{Raw data sources}\label{raw-data-sources}

The locations included in the data package comprise the 12 largest
(population-wise) Census metropolitan areas (CMA's) based on the 2016
Canadian Census \citep{governmentofcanada2016CensusPopulation2016}
\footnote{We included Oshawa, Ontario, as part of the Greater Toronto
  Area (GTA) because of the former's proximity to the latter. We also
  included Abbotsford-Mission, British Columbia, as part of the
  Vancouver metropolitan area because of the former's proximity to a
  transit station on the region's West Coast Express commuter rail line.}.
These locations are Toronto, Montreal, Vancouver, Ottawa-Gatineau,
Calgary, Edmonton, Quebec City, Winnipeg, Hamilton,
Kitchener-Cambridge-Waterloo, London, and Halifax. We used four main
data sources to construct the \{canaccesR\} data package: General
Transit Feed Specification (GTFS), OpenStreetMap (OSM), DMTI's Enhanced
Points of Interest, and Statistics Canada Census data.

We manually collected and processed the GTFS files from all transit
agencies within the selected CMA's to use their information on the
public transit schedule in 2019 and 2023. The OpenStreetMap data for the
selected areas were collected through the \{osmextract\} package
\citep{gilardiOsmextractDownloadImport2025}. We used OSM data from 2019
and 2023, which provided information on the areas' transit network in
two points in time. We collected data from the 2016 Canadian Census
using the \{cancensus\} package
\citep{vonbergmannCancensusPackageAccess2022} and used its information
on the spatial distribution of the population and the number of
workplace locations (employment) across the CMA's
\citep{governmentofcanada2016CensusPopulation2016}. Finally, we gathered
and cleaned the 2023 DMTI's Enhanced Points of Interest dataset to
obtain the location of the groceries stores within every urban area
selected \citep{dmtispatialincEnhancedPointsInterest2015}. We filtered
the locations within the DMTI dataset using the grocery stores code from
the North American Industry Classification System (NAICS) and the
Standard Industrial Classification (SIC). \{\textbf{THE CODE FOR THIS
ESTIMATIONS AND THE TRAVEL TIME MATRICES IS ON THE
transit\_death\_spiral github repo. 1) SHOULD WE CITE IT? 2) IS THAT A
PROBLEM?}\}

\subsection{Methods: travel time matrices
processing}\label{methods-travel-time-matrices-processing}

Using the \{r5r\} package, we estimated public transit travel times for
two destination types, grocery stores and jobs. For each amenity type,
we chose a likely travel time and day of the week. We set a 15 minutes
time window and the maximum trip duration to 120 minutes. The estimated
times are the median of the 15 minute time window. For groceries stores,
we set the departure date to a weekend afternoon and the departure time
to between 12:00 PM to 12:15 PM on April 20, 2019 and April 22, 2023.
For employment, we ran the analysis on a typical weekday morning
rush-hour commute, more specifically 8:00 to 8:15 AM departure on
Tuesday, April 16, 2019 and Tuesday, April 18, 2023 \footnote{The one
  exception is Quebec City, where the routing for 2019 occurs on a
  Saturday and Tuesday in June (instead of April) due to the GTFS data
  unavailability.}. In both cases, we assumed that walking was the mode
of travel from origin to transit stop and from transit stop to
destination. We aggregated all the resulting travel time matrices at the
Dissemination Area (DA) level, which comprise the fundamental unit of
analysis in data package.

\section{\{canaccessR\}'s contents}\label{canaccessrs-contents}

The package contains the following contents: travel time matrices,
socio-economic and demographic data disaggregated at the DA level, the
CMA areas' boundaries and backgrounds for plotting the data spatially,
and aggregated statistics.

The main contents of the \{canaccessR\} package are the travel time
matrices. These matrices (datasets) comprise the estimated travel time
by transit from and to each origin and destination Dissemination Area
pairs. The datasets also feature the total population in the origin DA
and the total employment in the destination DA. Besides these
information, the matrices contain the unique origin and destination DA
codes, region and name identifiers, the year in which the travel times
refers to, and the date and time of departure for a trip that originates
at the Dissemination Area of origin. There are two matrices for each
metropolitan area, one containing the travel times to jobs and the other
to groceries stores \footnote{The information from Toronto, Hamilton,
  and Waterloo are aggregated at the Greater Golden Horseshoe travel
  time matrices, making it 20 datasets in total.}. These information can
be turned into spatial data using the socio-economic and demographic
datasets.

The socio-economic and demographic data contain other information
disaggregated at the DA level for each CMA. These refer to total
population by age groups, number of dwellings, number of individuals
below the Low Income Measure, etc. The aggregated statistics refer to
population aggregates (for the selected CMA's) and transit revenue and
ridership data aggregated by regional and national scale.

Below, we present some descriptive statistics from the travel time
matrices contained in the \{canaccessR\} package.

\section{How to use \{canaccessR\}}\label{how-to-use-canaccessr}

This section presents some potential applications of the data package.

\section{Concluding remarks}\label{concluding-remarks}

In this paper, we describe the \{canaccesR\} data package, created using
the \{r5r\} package and transit schedule, street network, employment,
and population data. The package's main contents refers to the
ready-to-use travel time matrices for public transit to reach employment
and groceries stores in Canada's 12 largest urban areas. We expect the
contents of the package to be used in transportation accessibility
evaluations within and across those regions. Moreover, these datasets
can be used in further equity assessments that evaluate the distribution
of accessibility across space and between social groups. Furthermore, in
the spirit of open data products
\citep{arribas-belOpenDataProductsA2021}, the package can be expanded
through collaboration with other researchers by, for example, including
travel time matrices to other essential destinations within the DMTI's
dataset (\emph{e.g.}, schools, healthcare, etc.). In other words, we
hope that by making these datasets publicly available, future analysis
can contribute to making Canada's transportation system more just and
fair, considering accessibility's as the main social good of
transportation \citep{martensTransportJusticeDesigning2016}, and the
inherent connection between public transit and the ``right to the city''
\citep{cogginRightTransportMoving2015}.

\section{Declaration of Conflicting
Interests}\label{declaration-of-conflicting-interests}

The author(s) declared no potential conflicts of interest with respect
to the research, authorship, and/or publication of this article.

\section{Funding}\label{funding}

The author(s) disclosed receipt of the following financial support for
the research, authorship, and/or publication of this article: This work
was supported by the Social Sciences and Humanities Research Council of
Canada (\emph{More description about the funding source after the review
process}).

\section{ORCID}\label{orcid}

\begin{itemize}
\tightlist
\item
  name: João Pedro Figueira Amorim Parga orcid: 0000-0002-4105-5927
\item
  name: Anastasia Soukhov orcid: 0000-0003-4371-4831
\item
  name: Robert Nutifafa Arku orcid: 0000-0002-2018-886X
\item
  name: Christopher Higgins orcid: 0000-0002-3551-7750
\item
  name: Antonio Páez orcid: 0000-0001-6912-9919
\end{itemize}

\section{Data availability statement}\label{data-availability-statement}

The \{canaccessR\} data package can be found and installed on its Github
\href{https://github.com/paezha/canaccessR}{respository}.

\bibliographystyle{sageh}
\bibliography{bibfile.bib}


\end{document}
