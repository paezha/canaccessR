\documentclass[Royal,times,sageh]{sagej}

\usepackage{moreverb,url,natbib, multirow, tabularx}
\usepackage[colorlinks,bookmarksopen,bookmarksnumbered,citecolor=red,urlcolor=red]{hyperref}



% tightlist command for lists without linebreak
\providecommand{\tightlist}{%
  \setlength{\itemsep}{0pt}\setlength{\parskip}{0pt}}





\begin{document}


\setcitestyle{aysep={,}}

\title{canaccessR: An open data product for analyzing transportation
accessibility to employment and grocery stores in Canada's largest
metropolitan areas.}

\runninghead{}

\author{João Pedro Figueira Amorim Parga\affilnum{}, Anastasia
Soukhov\affilnum{}, Robert Nutifafa Arku\affilnum{}, Christopher
Higgins\affilnum{}, Antonio Páez\affilnum{}}

\affiliation{}



\begin{abstract}
In this paper, we describe the \{canaccesR\} package, an open data
product (ODPs) created using the R statistical language. \{canaccess\}
is a data package that provides public transit travel time estimates
(travel time matrices - TTM) to employment locations and grocery stores
across the 12 largest Canadian metropolitan areas. We calculate these
estimates for each Dissemination Area (DA) within these regions for the
years 2019 and 2023. To do so, we use the \{r5r\} R package, General
Transit Feed Specification (GTFS), OpenStreetMap (OSM), DMTI's Enhanced
Points of Interest, and Statistics Canada Census data. This data package
can be used by researchers, practitioners, and transit agencies to
estimate accessibility to essential services across these regions. These
estimates can be used to compare different regions across Canada in
terms of their accessibility and to conduct within-region equity
assessments regarding access to services, which can inform improvements
in transportation policies related to accessibility. The package is
still in its initial phase and may undergo expansions in the future by
adding TTM's for other destinations (e.g., schools, healthcare
facilities). Finally, as an ODPs, the \{canaccess\} package allows for
open exploration, use, and contribution by users through its GitHub
repository.
\end{abstract}

\keywords{Accessibility; public transit; open data products (ODPs);
travel time; employment; grocery stores.}

\maketitle

\section{Introduction}\label{introduction}

The objective of this paper is to describe the \{canaccesR\} open data
package. Its main contents are a set of public transit travel time
matrices (TTM) estimates to employment and groceries stores from the 12
largest Canadian metropolitan areas in the years 2019 and 2023. These
estimates were created by leveraging expertise in data science, computer
programming, and transportation accessibility, using the \{r5r\} R
package \citep{pereiraR5rRapidRealistic2021}. We used transport and
street network, population, and employment data from different sources
to estimate these TTM's datasets.

Estimating accessibility - the potential offered by the transportation
system to reach destinations
\citep{paezMeasuringAccessibilityPositive2012} - requires specialized
datasets and technical expertise. Despite recent advancements in the
field, ready-to-use available data on transportation accessibility is
still sparse. Thus, we leverage technical knowledge to make publicly
available ready-to-use data on transportation accessibility. To create
the package, we integrated and processed raw data from diverse sources,
estimated TTM's for two destinations types (e.g., jobs and groceries)
across the largest cities in the country, and distributed these findings
through this transparent and open source data product. Our main
contribution is to provide analysis-ready data for Canada's largest
cities, thus making the fields of accessibility research and urban
analytics more accessible.

The package's main audiences are Canadian researchers in urban planning
and transportation and transportation system agencies. We anticipate
three main uses for the open data product (ODP) described in this paper.
First, the datasets allow for static assessment of the level of public
transit accessibility across the country's largest cities before and
after the COVID-19 pandemic. In other words, we make it easier for those
interested to compare cities and their level of public transit
accessibility to essential destinations (such as employment centers and
groceries stores). Second, researchers can use the package's information
as inputs to evaluate changes in accessibility throughout this period.
Researchers may also investigate disparities in these changes across
space within those cities, given the spacial character of the datasets
made available here. Third, as is common practice in transportation
accessibility research, these evaluations can substantiate broader
investigations on matters of justice and equity in transportation. For
example, the TTM estimates allow for evaluating the evolution of public
transit's accessibility by income or spatial distribution across all
Dissemination Areas (DA's) of each of the 12 cities in the sample
\citep{pargaDemocraticAccessOur2024}.

Besides this introduction, we organize this paper as follows. The next
section contains a description of the data sources we used to construct
the data package. Then, we recount the data processing necessary to
create the package. Next, we go through the main contents of the data
package, i.e., the travel time matrices estimated through our analysis.
We present some basic descriptive statistics of these datasets, and
elucidate how one can use them in accessibility analysis. Finally, we
conclude by explaining how we expect \{canaccesR\} to contribute to the
urban analytics and science community.

\section{Data Sources}\label{data-sources}

The locations included in the data package comprise the 12 largest
(population-wise) Census metropolitan areas (CMA's) based on the 2021
Canadian Census \citep{governmentofcanada2021CensusPopulation2021}
\footnote{We included Oshawa, Ontario, as part of the Greater Toronto
  Area (GTA) because of the former's proximity to the latter. We also
  included Abbotsford-Mission, British Columbia, as part of the
  Vancouver metropolitan area because of the former's proximity to a
  transit station on the region's West Coast Express commuter rail line.},
which are Toronto, Montreal, Vancouver, Ottawa-Gatineau, Calgary,
Edmonton, Quebec City, Winnipeg, Hamilton, Kitchener-Cambridge-Waterloo,
London, and Halifax.

We used four main data sources to construct the \{canaccesR\} data
package: General Transit Feed Specification (GTFS), OpenStreetMap (OSM),
DMTI's Enhanced Points of Interest, and Statistics Canada Census data.

The GTFS files from the transit agencies from all the related areas
provided the information on public transit schedule and their changes
from 2019 to 2023.

The OpenStreetMap data provided information on the transit network and
was necessary for the routing used to estimate the travel time matrices.

The 2016 and 2021 Statistics' Canada Census data provided the
information on the spatial distribution of the population and the number
of workplace locations, which represent the quantity of employment
across the CMA's
\citep[governmentofcanada2021CensusPopulation2021]{governmentofcanada2016CensusPopulation2016}.
All of the resulting travel time matrices were aggregated at the
Dissemination Area (DA) level, which comprise the fundamental unit of
analysis in data package.

The DMTI's Enhanced Points of Interest

\section{Data processing}\label{data-processing}

Routing, cleaning, etc.

\section{\{canaccessR\}'s contents}\label{canaccessrs-contents}

The main contents of the \{canaccessR\} package are the travel time
matrices estimates for all the 12 largest Canadian cities.

Other sets of data are also available at the \{canaccessR\} package
besides the travel time matrices. These are the boundaries,
socio-economic and demographic data (e.g., population, number of
dwellings, number of individuals below the Low Income Measure, etc.) of
the selected CMA's, disaggregated by DA. In addition, the package also
contains aggregated population statistics (for the selected CMA's) and
transit revenue and ridership data aggregated by regional and national
scale.

\subsection{Descriptive statistics}\label{descriptive-statistics}

Below, we present some of the basic statistics of

\section{How to use \{canaccessR\}}\label{how-to-use-canaccessr}

This section presents some potential applications of the data package.

\section{Concluding remarks}\label{concluding-remarks}

In this paper, we describe the \{canaccesR\} data package, created using
the \{r5r\} R package and transit schedule, street network, employment,
and population data. The package's main contents refers to the
ready-to-use travel time matrices for public transit to reach employment
and groceries stores in Canada's 12 largest urban areas. We expect the
contents of the package to be used in transportation accessibility
evaluations within and across those regions. Moreover, these datasets
can be used in further equity assessments that evaluate the distribution
of accessibility across space and between social groups. In other words,
we hope that by making these datasets publicly available, future
analysis can contribute to making Canada's transportation system more
just and fair, considering accessibility's as the main social good of
transportation {[}MARTENS JUSTICE{]}, and the inherent connection
between public transit and the ``right to the city'' {[}COGGIN{]}.

\section{Declaration of Conflicting
Interests}\label{declaration-of-conflicting-interests}

The author(s) declared no potential conflicts of interest with respect
to the research, authorship, and/or publication of this article.

\section{Funding}\label{funding}

The author(s) disclosed receipt of the following financial support for
the research, authorship, and/or publication of this article: This work
was supported by the Social Sciences and Humanities Research Council of
Canada (\emph{More description about the funding source after the review
process}).

\section{ORCID}\label{orcid}

\begin{itemize}
\tightlist
\item
  name: João Pedro Figueira Amorim Parga orcid: 0000-0002-4105-5927
\item
  name: Anastasia Soukhov orcid: 0000-0003-4371-4831
\item
  name: Robert Nutifafa Arku orcid: 0000-0002-2018-886X
\item
  name: Christopher Higgins orcid: 0000-0002-3551-7750
\item
  name: Antonio Páez orcid: 0000-0001-6912-9919
\end{itemize}

\section{Data availability statement}\label{data-availability-statement}

The \{canaccessR\} data package can be found and installed on its Github
\href{https://github.com/paezha/canaccessR}{respository}.

\bibliographystyle{sageh}
\bibliography{bibfile.bib}


\end{document}
