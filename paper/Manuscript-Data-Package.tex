\documentclass[Royal,times,sageh]{sagej}

\usepackage{moreverb,url,natbib, multirow, tabularx}
\usepackage[colorlinks,bookmarksopen,bookmarksnumbered,citecolor=red,urlcolor=red]{hyperref}



% tightlist command for lists without linebreak
\providecommand{\tightlist}{%
  \setlength{\itemsep}{0pt}\setlength{\parskip}{0pt}}





\begin{document}


\setcitestyle{aysep={,}}

\title{canaccessR: An open data product for analyzing transportation
accessibility to employment and grocery stores in Canada's largest
metropolitan areas.}

\runninghead{}

\author{João Pedro Figueira Amorim Parga\affilnum{}, Anastasia
Soukhov\affilnum{}, Robert Nutifafa Arku\affilnum{}, Christopher
Higgins\affilnum{}, Antonio Páez\affilnum{}}

\affiliation{\affilnum{}{}}



\begin{abstract}
In this paper, we describe the \{canaccesR\} package, an open data
product (OPS) created using the R statistical language. \{canaccess\} is
a data package that provides public transit travel time estimates
(travel time matrices - TTM) to employment locations and grocery stores
across the 12 largest Canadian metropolitan areas. We calculate these
estimates for each Dissemination Area (DA) within these regions for the
years 2019 and 2023. To do so, we use the \{r5r\} R package, General
Transit Feed Specification (GTFS), OpenStreetMap (OSM), DMTI's Enhanced
Points of Interest, and Statistics Canada Census data. This data package
can be used by researchers, practitioners, and transit agencies to
estimate accessibility to essential services across these regions. These
estimates can be used to compare different regions across Canada in
terms of their accessibility and to conduct within-region equity
assessments regarding access to services, which can inform improvements
in transportation policies related to accessibility. The package is
still in its initial phase and may undergo expansions in the future by
adding TTM's for other destinations (e.g., schools, healthcare
facilities). Finally, as an OPS, the \{canaccess\} package allows for
open exploration, use, and contribution by users through its GitHub
repository.
\end{abstract}

\keywords{Accessibility; public transit; open data products; OPS; travel
time; employment; grocery stores.}

\maketitle

\section{Introduction}\label{introduction}

The objective of this paper is to describe the \{canaccesR\} Open Data
Product (OPS), an open data package. The package's main contents are
public transit travel time matrices (TTM) estimates to employment and
groceries stores from the 12 largest Canadian metropolitan areas in the
years 2019 and 2023. These estimates were created by leveraging
expertise in data science, computer programming, and transportation
accessibility, using the \{r5r\} R package {[}PEREIRA CITATION{]}. We
used transport and street network, population, and employment data from
different sources to arrive at our estimations.

The main contribution of this data package is to provide analysis-ready
data for Canada's largest cities, thus making the field of accessibility
research and urban analytics more accessible. Estimating accessibility,
i.e., the potential offered by the transportation system to reach
destinations {[}Páez, Scott, and Morency, ``Measuring
Accessibility.''{]} requires specialized datasets and technical
expertise. By integrating and processing raw data from diverse sources,
estimating TTM's for two destinations types (e.g., jobs and groceries)
across Canada's largest cities of the country, and making these findings
publicly available, we hope this OPS can advance the field of urban
analytics in Canada.

We anticipate that these datasets will enable researchers to analyze
changes in public transit accessibility to essential destinations, such
as employment centers and grocery stores, in the largest urban areas of
the country. Consequently, the package addresses the need for
ready-to-use public transit accessibility data, helping researchers,
transit agencies, and the general public in evaluating equity in
transportation accessibility within and across Canada.

Besides this introduction, we organize this paper as follows. The next
section contains a description of the data sources we used to construct
the data package. Then, we recount the data processing necessary to
create the package. Next, we go through the main contents of the data
package, i.e., the travel time matrices estimated through our analysis.
We present some basic descriptive statistics of these datasets, and
elucidate how one can use them in accessibility analysis. Finally, we
conclude by explaining how we expect \{canaccesR\} to contribute to the
urban analytics and science community.

\section{Data Sources}\label{data-sources}

We used four main data sources to construct the \{canaccesR\} data
package: General Transit Feed Specification (GTFS), OpenStreetMap (OSM),
DMTI's Enhanced Points of Interest, and Statistics Canada Census data.

\subsection{General Transit Feed
Specification}\label{general-transit-feed-specification}

\subsection{OpenStreetMaps}\label{openstreetmaps}

Files used for routing.

\subsection{Census Data}\label{census-data}

Employment etc.

\subsection{DMTI's Enhanced Points of
Interest}\label{dmtis-enhanced-points-of-interest}

\section{Data processing}\label{data-processing}

Routing, cleaning, etc.

\section{\{canaccessR\}'s contents}\label{canaccessrs-contents}

The main contents of the \{canaccessR\} package are the travel time
matrices estimates for all the 12 largest Canadian cities.

\subsection{Descriptive statistics}\label{descriptive-statistics}

Below, we present some of the basic statistics of the

\section{How to use \{canaccessR\}}\label{how-to-use-canaccessr}

This section presents some potential applications of the data package.

\section{Concluding remarks}\label{concluding-remarks}

In this paper, we describe the \{canaccesR\} data package, created using
the \{r5r\} R package and transit schedule, street network, employment,
and population data. The package's main contents refers to the
ready-to-use travel time matrices for public transit to reach employment
and groceries stores in Canada's 12 largest urban areas. We expect the
contents of the package to be used in transportation accessibility
evaluations within and across those regions. Moreover, these datasets
can be used in further equity assessments that evaluate the distribution
of accessibility across space and between social groups. In other words,
we hope that by making these datasets publicly available, future
analysis can contribute to making Canada's transportation system more
just and fair, considering accessibility's as the main social good of
transportation {[}MARTENS JUSTICE{]}, and the inherent connection
between public transit and the ``right to the city'' {[}COGGIN{]}.

\section{Declaration of Conflicting
Interests}\label{declaration-of-conflicting-interests}

The author(s) declared no potential conflicts of interest with respect
to the research, authorship, and/or publication of this article.

\section{Funding}\label{funding}

The author(s) disclosed receipt of the following financial support for
the research, authorship, and/or publication of this article: This work
was supported by the Social Sciences and Humanities Research Council of
Canada (\emph{More description about the funding source after the review
process}).

\section{ORCID}\label{orcid}

Author 1

Author 2

Author 3

\section{Data availability statement}\label{data-availability-statement}

The \{canaccessR\} data package can be found and installed on its Github
\href{https://github.com/paezha/canaccessR}{respository}.

\bibliographystyle{sageh}
\bibliography{bibfile.bib}


\end{document}
